\section{A4 Borsches Atommodell}
Atom = kleinstes Abbild des Planetensystems
\\
aber $F_C$ anstatt $F_G$ (elektrostat. Kraft $F_C$)$ = F_C = F_Z$
\\
\\
da elektrostat. Kraft bei gleichen $r \, 10^{39}$ mal stärkter als Gravitationskraft ist.
\\
\\
\mbox{\Large $\frac{1}{4 \pi \epsilon_0} \cdot \frac{e^2}{r^2} = m_e \cdot \frac{v^2}{r}$}
\\
\\
Problem: nach Gesetzen der Elektrodynamic müsste das Elektron abstrahlen (beschl. Ladung wg. Kreisbahn)
\\
\\
Energieverlust würde Elektron nach $10^{-16}s$ in den Kern stürzen lassen

\subsection{Bohrsche Postulate}

\begin{enumerate}
    \item die negativ geladenen Elektronen bewegen sich auf Kreisbahnen um den positiv geladenen Atomkern
    \item {\color{red} Der Drehimpuls is gequantelt gemäß \mbox{\Large $L = n \cdot \hbar$} Es gibt damit ausgezeichnete
    (stabile) Elektronenbahnen innerhalb eines Atoms, auf denen ein Elektron nicht abstrahlen muss.
    \\
    Erklärung in Wellenbild. Auf den stabilen Bahnen bildet sich eine stehende Welle aus $\,\to\,$ stabile Konfiguration $\,\to\,$ keine Abstrahlung}
    \item sofern ein Elektron von der Schale $m$ auf die Schale $n$ übergeht, aborbiert $(n<m)$ oder
    emmitiert $(n > m)$ es ein Photon entsprechend \\
    \mbox{\Large $ E_{nm} = \hbar \cdot \omega_{nm} = h \cdot f_{nm} = E_n - E_m$}
\end{enumerate}
