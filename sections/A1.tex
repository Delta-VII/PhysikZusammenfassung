\section{A.1 Der Teilchen-Dualismus - Welle beim Licht}

Kopuskeltheorie (Newton):\quad Licht ist ein Strom von Partikeln, die von der Lichtquelle ausgehend geradlinig den Raum durchqueren.\\

Wellentheorie (Huygens):\quad Licht ist eine Wellenbewegung des Äthers\\

Mit der Wellentheorie konnten alle optischen Phänomene des 17. - 19. Jhrs. erklärt werden.\\


Ende 19./Anfang 20. Jhrd wurden drei Phänomene festgestellt, die mit der Wellentheorie im Widerspruch standen:
\begin{itemize}
    \item Michelsonscher Interferenzversuch: Es gibt keinen Äther
    \item Lichtelektrischter Effekt:
        \subitem Die Energie der freigesetzten Elektronen ist proportional zur Frequenz des Lichtes, während die Zahl der freigesetzten Elektronen proportional zur Intensität des eingestrahlten Lichts ist.
        \subitem $\,\to\,$ Erklärung nur mit der Vorstellung von Photonen (Teilchen) verträglich

    \newpage
    \item Strahlungskurve des Schwarzen Strahlers
        \subitem Die spektrale Strahlungsdichte steigt mit $\lambda^5$ an, erreicht ein Maximum und fällt dann exponentiell zu langen Wellenlängen hin ab $\,\to\,$ erklärbar nur mit Photonen.
\end{itemize}

\begin{figure}[h]
    \centering
    \includegraphics[width=0.75\textwidth]{media/strahlungsdichte_schwarzer_körper.png}
    \caption{Strahlungsdichte am schwarzen Körper}
    \label{fig:meine-grafik}
   \end{figure}